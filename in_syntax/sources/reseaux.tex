% Syntaxe pour les r\'eseaux
% 5 novembre 2001



On utilise deux unit\'es lexicales pour d\'ecrire un r\'eseau
sous forme textuelle :

\begin{itemize}
%
\item Les \emph{symboles} servent \`a donner les cellules du r\'eseau.
%
\item Les \emph{variables} servent \`a nommer les ports du r\'eseau et
\`a sp\'ecifier les connexions entre eux.
%
\end{itemize}

% Il faut d'abord donner les cellules du r\'eseau.
% %
% On introduit des \emph{variables} afin de nommer le port principal et
% (\'eventuellement) les ports auxiliaires de chaque cellule.
% %

Voici un exemple de r\'eseau compos\'e de cellules
dont tous les ports sont connect\'es \`a des ports libres :
%
\cinput{../figures/misc/cell_net_syntax.tex}
%
Par convention, on d\'ecide de nommer les ports d'une cellule 
en suivant le sens horaire. Les parenth\`eses sont omises pour
les cellules d'arit\'e z\'ero.
%
Un r\'eseau peut contenir plusieurs cellules portant le m\^eme
symbole mais chaque port a un nom distinct ; ce qui supprime
toute ambiguit\'e.
%
On compl\`ete ensuite la description du r\'eseau en listant
les connexions entre les ports. Une connexion est donn\'ee par
un couple de variables. Une permutation $\varphi$ sur
$\{1, \ldots, n\}$ est d\'ecrite comme suit:
%
\cinput{../figures/misc/permutation.tex}
%
En composant la liste des cellules du r\'eseau et une permutation, on
peut ainsi d\'ecrire un r\'eseau quelconque:

\cinput{../figures/misc/misc_net_syntax.tex}

On distingue donc deux types de variables selon leurs nombres
d'occurrences dans la description d'un r\'eseau.
Les \emph{variables li\'ees} apparaissent deux fois et correspondent
\`a une connexion entre deux ports.
Les \emph{variables libres} apparaissent une fois et correspondent
\`a un port libre.
%
La syntaxe d'un r\'eseau est donc d\'ecrite par la grammaire suivante
o\`u les non-terminaux sont \'ecrits en majuscules :

\begin{center}
%
\textsf{
\begin{tabular}{l@{\ \ ::=\ \ }l}
%
R\'ESEAU	& vide $|$  COUPURE \  R\'ESEAU					\\
COUPURE		& TERME \connect\ TERME							\\
TERME		& variable $|$ symbole \ LISTE					\\
LISTE		& vide $|$ ( LISTE' )							\\
LISTE'		& TERME $|$ TERME , LISTE'
%
\end{tabular}
}
%
\end{center}

Remarquons que le non-terminal \textsf{COUPURE} correspond \`a une
coupure au sens usuel (entre deux cellules) ou \`a une connexion avec
un port libre.
%
Un r\'eseau peut \^etre sp\'ecifi\'e de plusieurs fa\c cons diff\'erentes.
Si $u$ et $v$ sont deux termes, on peut indiff\'eremment
\'ecrire $u \connect v$ ou $v \connect u$. De plus, on peut introduire
autant de variables li\'ees que l'on veut. Les descriptions suivantes
sont \'equivalentes :

\begin{center}
$
\begin{array}{llll}
%
\begin{array}[t]{l}
p \connect q \\
\end{array}
%
&
%
\begin{array}[t]{l}
p \connect x \\
x \connect q
\end{array}
%
&
%
\begin{array}[t]{l}
p \connect x \\
x \connect y \\
y \connect q
\end{array}
%
&
\cdots
\end{array}
$
\end{center}

On peut facilement condenser la description d'un r\'eseau.
Consid\'erons une variable li\'ee $x$ apparaissant dans une
coupure \(x \connect y\) o\`u $y$ est une variable ;
il suffit de supprimer la coupure \(x \connect y\) et
de remplacer l'autre occurrence de $x$ par $y$.
%
Plus g\'en\'eralement, on peut appliquer la m\^eme m\'ethode pour une
variable li\'ee apparaissant dans une coupure de la forme
$x \connect T$ o\`u $T$ est un terme.
%
En appliquant cette m\'ethode sur le r\'eseau pr\'ec\'edent,
on obtient une description condens\'ee sans coupure entre un terme
et une variable li\'ee ; on dit que le r\'eseau est \'ecrit selon
la syntaxe de Lafont \cite{Laf90}.

\cinput{../figures/misc/var_cut_elim.tex}

On peut aussi \'ecrire un r\'eseau en utilisant directement la
syntaxe de Lafont. Il suffit de repr\'esenter le r\'eseau
avec tous les ports principaux orient\'es vers le bas ; on
consid\`ere alors le r\'eseau comme une liste de couples d'arbres
auxquels on ajoute des liens entre les feuilles.
On introduit ainsi une variable li\'ee pour chaque lien axiome.
Un exemple est donn\'e dans la figure \ref{lafont_syntax:fig}.
%
La syntaxe condens\'ee de Lafont a \'evidemment l'avantage d'une
grande concision mais la syntaxe g\'en\'erale permet de composer
deux r\'eseaux en concat\'enant simplement leur description textuelle
et en ajoutant \'eventuellement des connexions entre leurs ports
libres.


\begin{figure}
\cinput{../figures/misc/lafont_syntax.tex}
\caption{\textit{Un r\'eseau \'ecrit selon la syntaxe de Lafont}}
\label{lafont_syntax:fig}
\end{figure}






