% syntaxe des r\`egles
% 15 novembre 2001


Les r\`egles d'interaction sont d\'ecrites en utilisant la m\^eme m\'ethode
que pour les r\'eseaux.
%
Il suffit de consid\'erer le r\'eseau obtenu en connectant
les ports libres du membre gauche et du membre droit selon leur
correspondance. On obtient un r\'eseau avec une coupure et aucun
port libre. La syntaxe de Lafont pour une r\`egle d'interaction
est donc :

\begin{center}
%
\textsf{
\begin{tabular}{l@{\ \ ::=\ \ }l}
%
R\`EGLE		& symbole \ LISTE \cut\ symbole \ LISTE
%
\end{tabular}
}
%
\end{center}

Les r\`egles d'un programme sont s\'epar\'ees par une virgule.
La figure \ref{combinators:fig} donne les r\`egles des combinateurs
(voir \cite{Laf95}) sous forme graphique et textuelle (selon la syntaxe
g\'en\'erale et selon la syntaxe de Lafont).

\begin{figure}[p]
\cinput{../figures/combinators/combinators_rules.tex}
\caption{\textit{Les r\`egles des combinateurs d'interaction}}
\label{combinators:fig}
\end{figure}





