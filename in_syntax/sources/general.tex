% syntaxe generale d'un programme
% 15 novembre 2001

On d\'ecrit la syntaxe concr\`ete des r\'eseaux d'interaction.
%
Un programme \emph{interactif} est compos\'e de trois parties
commen\c cant par les mots-cl\'es \texttt{symbol}, \texttt{rule}
et \texttt{net} :

\begin{tabular}{ll}
%
\texttt{symbol} &
d\'eclaration des symboles avec leur arit\'e respective.\\
%
\texttt{rule} &
liste des r\`egles d'interaction.\\
%
\texttt{net} &
r\'eseau sur lequel commence la r\'eduction.\\
%
\end{tabular}

\

On peut d\'eclarer un symbole de deux mani\`eres diff\'erentes :

\begin{center}
%
\textsf{
\begin{tabular}{l@{\ \ ::=\ \ }l}
%
DECLARATION & symbole \ : nombre $|$ symbole , couleur \ : nombre
%
\end{tabular}
}
%
\end{center}

La deuxi\`eme d\'eclaration permet de sp\'ecifier (en texte)
la couleur des cellules correspondantes.
%
Un m\^eme symbole peut \^etre d\'eclar\'e plusieurs fois avec
des arit\'es diff\'erentes ; l'interpr\`ete fera comme
s'il s'agissait de symboles diff\'erents.




